\chapter{Technologies}

Below the technologies and libraries used to develop the system:

\begin{itemize}
    \item \textbf{Protelis} $\rightarrow$ for the aggregate program
    \item \textbf{DingNet simulator} \footnote{\href{https://github.com/dimoibiehg/DingNet}{DingNet}}
    \item \textbf{MQTT} $\rightarrow$ for the communication between:
    \begin{itemize}
        \item LoRaWan network and Protelis backend
        \item Protelis nodes to exchange neighborhood information
        \item Protelis nodes and Neighborhood-Manager
    \end{itemize}
    \item \textbf{Kotlin} $\rightarrow$ to implement the Protelis backend
    \item \textbf{Java v.11} $\rightarrow$ to work on the DingNet simulator that is in Java
    \item \textbf{Eclipse Paho MQTT client library}\footnote{\href{https://github.com/eclipse/paho.mqtt.java}{MQTT library}} $\rightarrow$ to implement a MQTT client for a real MQTT broker
\end{itemize}

\section{DingNet simulator}
DingNet is a simulator for a LoRaWan network of class A. It allows to:
\begin{itemize}
    \item define different type of device
    \item configure the network parameter for each devices and gateways
    \item configure the position in the environment of devices and gateway
    \item add different kind of sensors to each device 
    \item define the path that a device has to follow
    \item simulate the communication between devices and gateways in both the directions
\end{itemize}

How it works:
\begin{itemize}
    \item the simulator is single thread
    \item the simulator is "time-based" with a global clock that schedule all the computation
    \item every device send a packet every X seconds. If the packet is the same of the previous one, the packet isn't sent to avoid use bandwidth for useless message
    \item if a device don't send any message for Y seconds (with Y $>$ X) then the device send a "keepAlive" message
    \item when a gateway receive a message from a device, it publish the message on MQTT. Then it check if the there is a message to send to the device and if present send it.
    \item a device can receive only one message after every sended message.
    \item during a simulation step:
    \begin{itemize}
        \item device consume packet, if it has one
        \item every movable device moves of one meter if it is pass enough time according to the movement speed
        \item at the end the global clock is increment of one tick (now 1 millisecond)
    \end{itemize}
    \item when the global clock is incremented, it check if are present some trigger/event??? to fire. Now the events??? present are event to send packets and events to know when a packet is arrived to destination. In future they will be used also for:
    \begin{itemize}
        \item mote movement
        \item assure that a mote can't send more message at the same time
        \item ALOHA protocol
    \end{itemize}
\end{itemize}