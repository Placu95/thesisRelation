\chapter{Requirement}

The aim of the thesis is build a simulation for pervasive system for smart-city, where we want use the Aggregate Computing (AC) paradigm over a LoRaWan network. The simulation should be enough realistic to allow to switch to a real deploy with a small effort.

\section{Case study}

We want to define an application able to provide to the user the healthiest route to reach a destination. The route generation will be based on air quality level of the areas to across to reach the destination. The user -- in order to obtain the route -- has to require it to the application deployed in a remote server.

The system should use real-time data to create a city map of quality air.
The application will use the map to define a route to a destination that avoid areas with poor air quality.
If the air quality condition change, the application have to recompute the best route and notify it to the user. 
Data can be produced by two types of sensors:
\begin{itemize}
    \item \textbf{Fixed}: positioned along the roads and at intersections
    \item \textbf{Mobile}: placed on public transport or bicycles
\end{itemize}

All the sensors will be deployed in the LoRaWan network and they will use the LoRa technology to communicate their sensed value. Similarly also the user device, that will interact with the application to require the route, will use the LoRa technology.